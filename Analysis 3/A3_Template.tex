\documentclass[a4paper,12pt]{article}
\usepackage[left = 1.5cm, top = 2cm, right = 1.5cm, bottom = 2cm]{geometry}
\usepackage{amssymb,amsmath,graphicx,color,verbatim,fancyhdr,bm}

\usepackage{caption}
\usepackage{subcaption}
\usepackage{float}

\setlength\parindent{0pt}

%\pagecolor{black}
%\color{white}

\newcommand{\pencil}{\includegraphics[width = 0.025\textwidth]{pencilemoji.png}}

\newcommand{\computer}{\includegraphics[width = 0.025\textwidth]{computeremoji.png}}

\begin{document}



\textbf{Assignment 3 \LaTeX Template}\bigskip

\textbf{Important}: In Analyses 1e and 3b the template provides space for a single image, as we recommend generating your 4 plots using \texttt{par(mfrow = c(2, 2))}. If you generate four separate plots these should all be included.\bigskip

\textbf{Analysis 1}\bigskip

\textbf{1a}: My ID number is [12345678].\bigskip

\textbf{1b}: I [do/do not] have concerns about measurement error in the \texttt{first.tweet} variate. This is because [brief justification].\bigskip

\textbf{1c}:

\begin{center}
\begin{tabular}{|r | r | r | r | r | r | r| }\hline
 & Sample & Sample & Sample & Sample & Sample & Sample \\
 & Size & Mean & Median & Minimum & Maximum & SD\\\hline
Tweet Set A &   &   &   &   &   &   \\
Tweet Set B &   &   &   &   &   &   \\ \hline
\end{tabular}
\end{center}

\textbf{1d}: The maximum value of \texttt{tweet.gap.hour} for Tweet Set B should not be greater than 24 because [brief explanation].\bigskip

\textbf{1e}:

\begin{figure}[H]
     \centering
     \includegraphics[width = 0.5\textwidth]{goose2.jpeg}
\end{figure}




\textbf{1f}: \textbf{Tweet Set A}: Based on the results in Analysis [1c/1e], we can see that [...] while for data generated from an Exponential distribution we would expect to see [...]. [Add more comparisons here.] Overall, the Exponential model [description of whether the model fits well].\bigskip

\textbf{Tweet Set B}: Based on the results in Analysis [1c/1e], we can see that [...] while for data generated from an Exponential distribution we would expect to see [...]. [Add more comparisons here.] Overall, the Exponential model [description of whether the model fits well].

Overall, the Exponential model appears to be a better fit for [Tweet Set A/Tweet Set B], because [brief justification].\bigskip



\newpage

\textbf{Analysis 2}\bigskip

\textbf{2a}: My ID number is [12345678].\bigskip

\textbf{2b}: The maximum likelihood estimate of $\theta$ based on my sample is [number].\bigskip

\textbf{2c}: Relative Likelihood Function Plot:

\begin{figure}[H]
     \centering
     \includegraphics[width = 0.5\textwidth]{goose5.jpeg}
\end{figure}

\textbf{2d}: The 15\% likelihood interval for $\theta$ is [lower bound, upper bound].\bigskip

\textbf{2e}: The approximate 15\%, 95\% and 99\% confidence intervals for $\theta$ are [15\% lower bound, 15\% upper bound], [95\% lower bound, 95\% upper bound], and [99\% lower bound, 99\% upper bound], respectively. These were calculated by [explanation].\bigskip

\textbf{2f}: The approximate [15\%/95\%/99\%] confidence interval is most similar to the 15\% likelihood interval. This [is/is not] what I would expect, because [brief justification].\bigskip

\textbf{2g}: The interval [95\% lower bound, 95\% upper bound] tells us that [answer continues].\bigskip

%\textbf{2h}: This does not have to be written in complete sentences, but an example of how you could phrase your answer:\bigskip

%``An approximate 95\% likelihood interval for $\theta$ based on the likelihood ratio statistic is [lower bound, upper bound].''

\newpage

\textbf{Analysis 3}\bigskip

\textbf{3a}: My ID number is [12345678].\bigskip

\textbf{3b}: 

\begin{figure}[H]
     \centering
     \includegraphics[width = 0.5\textwidth]{goose7.jpeg}
\end{figure}

\textbf{3c}: The Gaussian model appears to fit the [Square Root/Log/Reciprocal] transformed data best, because [brief justification].

\newpage

\textbf{Analysis 4}\bigskip

\textbf{4a}: My ID number is [12345678]. In Analysis 3c I chose the [Square Root/Log/Reciprocal] transformation.\bigskip

\textbf{4b}: The sample size is [value], the sample mean is [value], the sample standard deviation is [value].\bigskip

\textbf{4c}: A 95\% [confidence interval/approximate confidence interval] for $\mu$ is [lower bound, upper bound]. This was calculated by [explanation].\bigskip

\textbf{4d}: This is an [exact/approximate] confidence interval, because [brief justification].\bigskip

\textbf{4e}: The interval [95\% lower bound, 95\% upper bound] tells us that [answer continues].\bigskip

\textbf{4f}: A 95\% confidence interval for $\sigma$ is [lower bound, upper bound]. This was calculated by [explanation].\bigskip

\textbf{4g}: I would conclude [conclusion] about the width of Alex's confidence intervals compared to those I calculated in Analysis 4f. This is because [brief justification].


\end{document}



